\documentclass[19pt]{article}
\usepackage[utf8]{inputenc}
\usepackage{amsmath}

% partial derivative as a fraction
\newcommand{\fracpd}[2]{
  \ensuremath{\frac{\partial #1}{\partial #2}}
}

% fraction with parenthesis around it
\newcommand{\pfrac}[2]{
  \ensuremath{ \left( \frac{#1}{#2} \right)}
}



\begin{document}

\section{Model}

\textit{About notation} Without loss of generality when I write $x_t$ I really mean $x(s^t)$. When the history dependence is potentially unclear, I will it out in longer form.

The Lagrangian is

\begin{alignat}{2} \label{eq:lagrangian}
  \mathcal{L} &= \mu_{1,0} V_{1,0}\left(c_{1,0}, V_{1,1}\right) &+& \mu_{2,0} V_{2,0}\left(c_{2,0}, V_{2,1} \right) \notag \\
  & & \sum_{t=0}^{\infty} \sum_{s^t}  \biggm \{
      &\lambda_{1}(s^t) \left( Y_1\left(\chi_{1,t}/g_t, 1\right) - a_{1,t} - a_{2,t} - i_{1,a,t} - i_{2,a,t} \right) + \notag \\
  & & &\lambda_{2}(s^t) \left( Y_2\left(\chi_{2,t}/g_t, \xi_t\right) - b_{1,t} - b_{2,t} - i_{1,b,t} - i_{2,b,t} \right) + \notag \\
  & & &\lambda_{3}(s^t) \left( (1-\delta_1)\frac{\chi_{1,t-1}}{g_{t-1}} + H_1 \left(i_{1,a,t-1}, i_{1,b,t-1} \right) - \chi_{1,t} \right) + \notag \\
  & & &\lambda_{4}(s^t) \left( (1-\delta_2)\frac{\chi_{2,t-1}}{g_{t-1}} + H_2 \left(i_{2,a,t-1}, i_{2,b,t-1} \right) - \chi_{2,t} \right)  \biggm\},
\end{alignat}

where $c_{i,t} = G_i(a_{i,t}, b_{i,t})$ and $\chi_{i,t} = g(s^t) k_i(s^t)$ for $i=1,2$ and $t=0, 1, \dots, \infty$.

\section{FOC}

The control variables are: $\left\{a_{n}(s^t), b_{n}(s^t), i_{n,a}(s^t), i_{n,b}(s^t), \chi_{n}(s^t) \right\}$ for $t=0,1,\dots, \infty$, $n \in {1,2}$, and all histories of states $s^t$. The FONC are

\begin{alignat}{2}
  a_{1}(s^t)   : \quad & \mu_{1,0} \fracpd{V_{1,0}}{V_{1,t}}\fracpd{V_{1,t}}{c_{1,t}} \fracpd{c_{1,t}}{a_{t,1}} &= \lambda_{1}(s^t) \label{eq:foc_a1} \\
  b_{1}(s^t)   : \quad & \mu_{1,0} \fracpd{V_{1,0}}{V_{1,t}}\fracpd{V_{1,t}}{c_{1,t}} \fracpd{c_{1,t}}{b_{t,1}} &= \lambda_{2}(s^t) \label{eq:foc_b1} \\
  i_{1,a}(s^t) : \quad & \sum_{s^{t+1}} \lambda_3(s^{t+1}) \fracpd{H_1(i_{1,a}(s^t),i_{1,b}(s^t)))}{i_{1,a}(s^t)} &= \lambda_1(s^t) \label{eq:foc_i1a} \\
  i_{1,b}(s^t) : \quad & \sum_{s^{t+1}} \lambda_3(s^{t+1}) \fracpd{H_1(i_{1,a}(s^t),i_{1,b}(s^t)))}{i_{1,b}(s^t)} &= \lambda_2(s^t)\label{eq:foc_i1b} \\
  \chi_{1}(s^t): \quad & \lambda_1(s^t) \fracpd{Y_1(\chi_{1,t}/g_t, 1)}{(\chi_{1,t}/g_t)} \frac{1}{g_t} + \sum_{s^{t+1}} \frac{(1 - \delta_1)}{g(s^{t+1}|s^t)}\lambda_3(s^{t+1}) &= \lambda_3(s^t) \label{eq:foc_chi1} \\
  a_{2}(s^t)   : \quad & \mu_{2,0} \fracpd{V_{2,0}}{V_{2,t}}\fracpd{V_{2,t}}{c_{2,t}} \fracpd{c_{2,t}}{a_{t,2}} &= \lambda_{1}(s^t)\label{eq:foc_a2} \\
  a_{2}(s^t)   : \quad & \mu_{2,0} \fracpd{V_{2,0}}{V_{2,t}}\fracpd{V_{2,t}}{c_{2,t}} \fracpd{c_{2,t}}{b_{t,2}} &= \lambda_{2}(s^t)\label{eq:foc_b2} \\
  i_{2,a}(s^t) : \quad & \sum_{s^{t+1}} \lambda_4(s^{t+1}) \fracpd{H_2(i_{2,a}(s^t),i_{2,b}(s^t)))}{i_{4,a}(s^t)} &= \lambda_1(s^t) \label{eq:foc_i2a} \\
  i_{2,b}(s^t) : \quad & \sum_{s^{t+1}} \lambda_4(s^{t+1}) \fracpd{H_2(i_{2,a}(s^t),i_{2,b}(s^t)))}{i_{4,b}(s^t)} &= \lambda_2(s^t)\label{eq:foc_i2b} \\
  \chi_{2}(s^t): \quad & \lambda_2(s^t) \fracpd{Y_2(\chi_{2,t}/g_t, 1)}{(\chi_{2,t}/g_t)} \frac{1}{g_t} + \sum_{s^{t+1}} \frac{(1 - \delta_1)}{g(s^{t+1}|s^t)}\lambda_4(s^{t+1}) &= \lambda_4(s^t) \label{eq:foc_chi2}
\end{alignat}

\section{Algebra} \label{sec:algebra}

  \subsection{Definitions} \label{sub:definitions}

  Before proceeding, we will gather some tools we will use while analyzing the above conditions. First, notice that by the chain rule we have

  \begin{align} \label{eq:chain_derivs}
    \fracpd{V_{i,0}}{V_{i,t}} = \prod_{\tau=1}^{t} \fracpd{V_{i,\tau-1}}{V_{i,\tau}}.
  \end{align}

  The notice that we can write the SDF for agent $i$ as

  \begin{align*}
    M_{i,t+1} &= \fracpd{V_{i,t}}{V_{i,t+1}} \frac{\partial V_{i,t+1}/ \partial c_{i,t+1}}{\partial V_{i,t}/ \partial c_{i,t}} \\
    &= \beta \left( \frac{C_{t+1}}{C_t} \right)^{\rho - 1} \left( \frac{V_{i,t+1}}{E \left[ V_{i,t+1}^{\alpha} \right]^{1/\alpha}} \right)^{\alpha-\rho}.
  \end{align*}

  We will also write stochastic discount factor in $a$ and $b$ units as follows:



  We will make use of the time $t$ Pareto weight, which we define as

  \begin{align} \label{eq:defn_pareto_weight}
    \mu_{i,t} &= \mu_{i,0} \fracpd{V_{i,0}}{V_{i,t}} \fracpd{V_{i,t}}{c_{i,t}} \notag \\
    &= \mu_{i,0} \prod_{\tau=1}^{t} \fracpd{V_{i,\tau-1}}{V_{i,\tau}} \fracpd{V_{i,t}}{c_{i,t}} \notag  \\
    &= \mu_{i,t-1} \fracpd{V_{i,t}}{V_{i,t+1}} \frac{\partial V_{i,t+1}/ \partial c_{i,t+1}}{\partial V_{i,t}/ \partial c_{i,t}} \notag \\
    &= \mu_{i, t-1} M_{i,t}
  \end{align}

  We will also work with the variable, $S_t = \frac{\mu_{1,t}}{\mu_{2,t}}$, which according to \eqref{eq:defn_pareto_weight} can be written recursively as

  \begin{align*}
    S_t &= S_{t-1} \frac{M_{1,t-1}}{M_{2,t-1}}
  \end{align*}

  \subsection{Working with FOC} \label{sub:Working_with_foc}


  We can take \eqref{eq:foc_i1a} and turn it into

  \begin{align*}
    \frac{1}{\fracpd{H_1(i_{1, a}(s^t), i_{1, b}(s^t)}{i_{1, a}(s^t)}} &= E \left[ M_{1, t+1} P_{k, t+1} \right]
  \end{align*}

  We can also transform \eqref{eq:foc_chi1} to

  \begin{align*}
    P_{k, t} &= \fracpd{Y(\chi_{1, t} / g(s^t))}{\chi_{1, t} / g(s^t)} \frac{1}{g(s^t)} + E \left[ \frac{(1 - \delta)}{g(s^{t+1})} M_{1, t+1} P_{k, t+1} \right]
  \end{align*}

  where $P_{k, t} = \frac{\lambda_{3, t}}{\lambda_{1, t}}$ and can be interpreted as the price of capital in terms of good $a$.

  \section{algebra}

  Using \eqref{eq:defn_pareto_weight}, \eqref{eq:foc_a1}, and \eqref{eq:foc_a2} we can derive

  \begin{align} \label{eq:eqn_for_a}
    \mu_{1,t} \fracpd{c_{1,t}}{a_{1,t}} &= \mu_{2,t} \fracpd{c_{2,t}}{a_{2,t}} \notag \\
    S_{t} \fracpd{c_{1,t}}{a_{1,t}} &= \fracpd{c_{2,t}}{a_{2,t}}
  \end{align}

  Similarly using \eqref{eq:defn_pareto_weight}, \eqref{eq:foc_b1}, and \eqref{eq:foc_b2} we can derive

  \begin{align} \label{eq:eqn_for_b}
    \mu_{1,t} \fracpd{c_{1,t}}{b_{1,t}} &= \mu_{2,t} \fracpd{c_{2,t}}{b_{2,t}} \notag \\
    S_{t} \fracpd{c_{1,t}}{b_{1,t}} &= \fracpd{c_{2,t}}{b_{2,t}}
  \end{align}



%
% The state variables
%
% We can combine \eqref{eq:foc_c1} and \eqref{eq:foc_a1} to obtain
%
% \begin{align} \label{eq:no_lambda1_1}
%   \mu_{1,0} \fracpd{V_{1,0}}{V_{1,t}} \fracpd{V_{1,t}}{c_{1,t}} \fracpd{G_1(a_{1,t}, b_{1,t})}{a_{1,t}} = \lambda_{3,t}.
% \end{align}
%
% Similarly, \eqref{eq:foc_c2} and \eqref{eq:foc_a2} give
%
% \begin{align} \label{eq:no_lambda1_2}
%   \mu_{2,0} \fracpd{V_{2,0}}{V_{2,t}} \fracpd{V_{2,t}}{c_{2,t}} \fracpd{G_2(a_{2,t}, b_{2,t})}{a_{2,t}} = \lambda_{3,t}.
% \end{align}
%
% Together \eqref{eq:no_lambda1_1} and \eqref{eq:no_lambda1_2} give
%
% \begin{align} \label{eq:mu0_ratio}
%   \frac{\mu_{1,0}}{\mu_{2,0}} &= \frac{\fracpd{V_{2,0}}{V_{2,t}} \fracpd{V_{2,t}}{c_{2,t}} \fracpd{G_2(a_{2,t}, b_{2,t})}{a_{2,t}}}{\fracpd{V_{1,0}}{V_{1,t}} \fracpd{V_{1,t}}{c_{1,t}} \fracpd{G_1(a_{1,t}, b_{1,t})}{a_{1,t}}}.
% \end{align}
%
% Now we define an ad hoc time $t$ Pareto weight:
%
% \begin{align} \label{eq:mu_t}
%   \mu_{i,t} &= \mu_{i,0} \fracpd{V_{i,0}}{V_{i,t}} \fracpd{V_{i,t}}{c_{i,t}} c_{i,t} \notag \\
%   &= \mu_{i,t-1} \fracpd{V_{i,t-1}}{V_{i,t}} \frac{\partial V_{i,t}/ \partial c_{i,t}}{\partial V_{i,t-1}/ \partial c_{i,t-1}} \frac{c_{i,t}}{c_{i,t-1}} \\
%   &= \mu_{i,t-1} M_t \frac{c_{i,t}}{c_{i,t-1}}
% \end{align}
%
% where we have used the chain rule to write $\fracpd{V_{i,0}}{V_{i,t}} = \prod_{\tau=1}^{t} \fracpd{V_{i,\tau-1}}{V_{i,\tau}}$ and have used
%
% \begin{align} \label{eq:MRS}
%   M_{i,t+1} &= \fracpd{V_{i,t}}{V_{i,t+1}} \frac{\partial V_{i,t+1}/ \partial c_{i,t+1}}{\partial V_{i,t}/ \partial c_{i,t}} \\
%   &= \beta \left( \frac{c_{i,t+1}}{c_{i,t}}\right)^{\rho - 1} \left( \frac{V_{i,t+1}}{E \left[ V_{i,t+1}^{\alpha} \right]^{\alpha}} \right)^{\alpha - \rho}
% \end{align}
%
% We can combine \eqref{eq:mu_t} with \eqref{eq:mu0_ratio} to get
%
% \begin{align} \label{eq:foobar}
%   \frac{1}{c_{1,t}} \mu_{1,t} \fracpd{G_{1,t}(a_{1,t}, b_{1,t})}{a_{1,t}} = \frac{1}{c_{2,t}} \mu_{2,t} \fracpd{G_{2,t}(a_{2,t}, b_{2,t})}{a_{2,t}}
% \end{align}
%
% Define $S_t = \frac{\mu_{1,t}}{\mu_{2,t}}$ and write \eqref{eq:foobar} as
%
% \begin{align} \label{eq:a_allocation}
%   S_t \fracpd{G_{1,t}(a_{1,t}, b_{1,t})}{a_{1,t}} \frac{1}{c_{1,t}} &= \fracpd{G_{2,t}(a_{2,t}, b_{2,t})}{a_{2,t}} \frac{1}{c_{2,t}}.
% \end{align}
%
% We can also use \eqref{eq:mu_t} for both agents to write:
%
% \begin{align} \label{eq:S_recursion}
%   S_t = S_{t-1} \frac{M_{1,t}}{M_{2,t}} \frac{c_{1,t}/c_{1,t-1}}{c_{2,t}/c_{2,t-1}}.
% \end{align}
%
% Following the same steps we can write the marginal condition for good $b$ as
%
% \begin{align} \label{eq:b_allocation}
%   S_t \fracpd{G_{1,t}(a_{1,t}, b_{1,t})}{b_{1,t}} \frac{1}{c_{1,t}} &= \fracpd{G_{2,t}(a_{2,t}, b_{2,t})}{b_{2,t}} \frac{1}{c_{2,t}}.
% \end{align}
%
% \subsection{Investment terms} \label{sub:Investment terms}
%
% TODO: determine why They write C5 and C6 lagged one period. Do we need to do this also? If so, why. If not, why not?




\end{document}
